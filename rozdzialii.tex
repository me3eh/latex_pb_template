\chapter[Inny tytuł do spisu treści]{Inne przykłady}
\section{Cytowania}
Literature cytujemy przez $\backslash cite\{nazwa1,nazwa2\}$ przykładowo $\backslash cite{NagraTC02}$ da w efekcie \cite{NagraTC02} lub $\backslash cite\{NagraTC02,iso9126\}$ - \cite{NagraTC02,iso9126}.
\section{Wypunktowania}
Wypunktowanie stosujemy 
\begin{verbatim}
    \begin{enumerate}
    \item pierwsze
    \item drugie
    \end{enumerate}
\end{verbatim}
co daje efekt jako:
    \begin{enumerate}
    \item pierwsze
    \item drugie
    \end{enumerate}
lub też jako
\begin{verbatim}
    \begin{itemize}
    \item jeden
    \item dwa
    \end{itemize}
\end{verbatim}
co daje efekt jako:
    \begin{itemize}
    \item jeden
    \item dwa
    \end{itemize}
\subsection{Wypunktowania mieszane}
\begin{verbatim}
\begin{enumerate}
    \item 1
    \begin{itemize}
        \item 1.1
        \item 1.2
    \end{itemize}
    \item 2
    \begin{itemize}
        \item 2.1
        \item 2.2
\end{itemize}
\end{enumerate}
\end{verbatim}
efekt kńcowy
\begin{enumerate}
    \item 1
    \begin{itemize}
        \item 1.1
        \item 1.2
    \end{itemize}
    \item 2
    \begin{itemize}
        \item 2.1
        \item 2.2
\end{itemize}
\end{enumerate}
\section{Tabele}
Tabele wstawiamy przez
\begin{verbatim}
\begin{table}[t]
\centering
\begin{tabular}{|ccc|}%rodzaj kolumn
\hline
1 kolumna & 2 kolumna & 3 kolumna \\
\hline
\end{tabular}
\caption{Opis tabeli}
\label{tab:p1}%referencja
\end{table}
\end{verbatim}
\begin{table}[t]
\centering
\begin{tabular}{|ccc|}%rodzaj kolumn
\hline
1 kolumna & 2 kolumna & 3 kolumna \\
\hline
\end{tabular}
\caption{Opis tabeli}
\label{tab:p1}%referencja
\end{table}
Do tabeli odwołujemy się przez
